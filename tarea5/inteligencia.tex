\documentclass[11pt,a4paper]{article}
\usepackage[utf8]{inputenc}
\usepackage[activeacute,spanish]{babel}
\usepackage{graphicx}
\author{Miguel Angel Aguirre Olvera}
\title{Tarea 5 Inteligencia artificial}
\date{}
\usepackage{multicol}
\DeclareGraphicsExtensions{.png}
\begin{document}
\maketitle 
\begin{multicols}{2}
\begin{center}
\textbf{como surgio la inteligencia artificial}
\end{center}
En 1950, el científico Alan Turing se preguntaba si las máquinas (calculadoras y computadoras básicamente) eran capaces de pensar.
Turing se preocupaba principalmente por establecer los límites y las diferencias de la inteligencia natural y la inteligencia artificial, y aunque el término como tal se aceptó hasta 1956, Turing diseñó la primera computadora capaz de jugar ajedrez.
Al fallecer prematuramente siendo juzgado por su sexualidad, el científico Von Neumann continuó su trabajo. Creía que las computadoras debían diseñarse a partir de un modelo del cerebro humano.
Sin embargo, posteriormente se descubrió que era mejor estudiar las funciones del cerebro para saber cómo desarrollar una máquina que pudiera realizar las mismas. Que en lugar de crear una máquina similar a nivel celular, fuera similar en la forma de procesar la información.
Basándose en el modelo de Turing, comenzó desarrollándose una inteligencia capaz de resolver juegos (como las damas y el ajedrez) que tuviera un gran número de situaciones por calcular, problemas a solucionar, tomar decisiones, hacer memoria, corregir los errores, entre otros.
Si bien las computadoras son capaces de responder a estos estímulos, no significa que los comprendan.
El término hoy en día se utiliza para añadirlo como adjetivo a todo aquello que tiene una inteligencia similar a la de los humanos.
\begin{center}
\textbf{tipos de inteligencia artificial}
\end{center}
Sistemas que imitan el funcionamiento del sistema nervioso por medio de redes neuronales artificiales. Este tipo de inteligencia automatiza la toma de decisiones, resolución de problemas, y el aprendizaje.
Sistemas que imitan el comportamiento físico del hombre (androides). La meta es que los robots realicen tareas de manera más eficiente que los humanos.
Sistemas que imitan el pensamiento lógico de los humanos, es decir, que perciben, razonan, y actúan.
Sistemas que actúan de manera racional, es decir, que son capaces de percibir el entorno y actuar en consecuencia.
Actualmente –sobre todo en la red social Telegram –se ha implementado el uso de chatbots, es decir sistemas de Inteligencia Artificial que “conversan” contigo. Pocas veces los humanos han notado la diferencia entre platicar con un sistema y con otro ser humano. Este experimento cumple un concepto hecho por Alan Turing, en el que afirmaba que la Inteligencia Artificial existiría como tal hasta que no fuéramos capaces de discernir una diferencia entre ambos.
\begin{center}
\textbf{inteligencia artificial otro enfoque}
\end{center}

El termino inteligencia artificial fue acuñado formalmente en 1956 durante la conferencia de Darthmounth, más para entonces ya se había estado trabajando en ello durante cinco años en los cuales se había propuesto muchas definiciones distintas que en ningún caso habían logrado ser aceptadas totalmente por la comunidad investigadora. La AI es una de las disciplinas más nuevas que junto con la genética moderna es el campo en que la mayoría de los científicos más les gustaría trabajar.

Una de las grandes razones por la cuales se realiza el estudio de la IA es él poder aprender más acerca de nosotros mismos y a diferencia de la psicología y de la filosofía que también centran su estudio de la inteligencia, IA y sus esfuerzos por comprender este fenómeno están encaminados tanto a la construcción de entidades de inteligentes como su comprensión.
La llegada de las computadoras a principios de los 50, permitió el abordaje sin especulación de estas facultades mentales mediante una autentica disciplina teórica experimental. Es a partir de esto que se encontró que la IA constituye algo mucho más complejo de lo que se pudo llegar a imaginar en principio ya que las ideas modernas que constituyen esta disciplina se caracterizan por su gran riqueza, sutileza e interés; en la actualidad la IA abarca una enorme cantidad de subcampos que van desde áreas de propósito general hasta tareas especificas.

Cuando se aplican algoritmos a la solución de los problemas aunque no se está actuando inteligentemente si esta siendo eficaz pero los problemas realmente complicados a los que se enfrenta el ser humano son aquellos en los cuales no existe algoritmo conocido así que surgen de reglas que tratan de orientarnos hacia las soluciones llamadas Heurísticas en las cuales nunca nada nos garantiza que la aplicación de una de estas reglas nos acerque a la solución como ocurre con los anteriores.

A partir de estos datos; Farid Fleifel Tapia describe a la IA como: "la rama de la ciencia de la computación que estudia la resolución de problemas no algorítmicos mediante el uso de cualquier técnica de computación disponible, sin tener en cuenta la forma de razonamiento subyacente a los métodos que se apliquen para lograr esa resolución.


\end{multicols}
\end{document}


