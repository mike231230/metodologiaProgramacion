\documentclass[11pt,a4paper]{article}
\usepackage[utf8]{inputenc}
\usepackage[activeacute,spanish]{babel}
\usepackage{graphicx}
\author{Miguel Angel Aguirre Olvera}
\title{Tarea 10 Java }
\usepackage{multicol}

\begin{document}
\maketitle 
\begin{multicols}{2}
\begin{center}
\textbf{Historia de lenguaje java }
\end{center}
Java es un lenguaje de programación orientado a objetos desarrollado por Sun Microsystems a principio de los años 90s.

En Diciembre de 1950 Patrick Naughton, ingeniero de Sun Microsystems, reclutó a varios colegas entre ellos James Gosling y Mike Sheridan para trabajar sobre un nuevo proyecto conocido como "El proyecto verde".
Con la ayuda de otros ingenieros, empezaron a trabajar en una pequeña oficina en Sand Hill Road en Menlo Park, California. Y así interrumpió todas las comunicaciones regulares con Sun y trabajó sin descanso durante 18 meses.

Intentaban desarrollar una nueva tecnología para programar la siguiente generación de dispositivos inteligentes, en los que Sun veía un campo nuevo a explorar. Crear un lenguaje de programación fácil de aprender y de usar. 
En un principio se consideraba C++ como lenguaje a utilizar, pero tanto Gosling como Bill Joy lo encontraron inadecuado. Gosling intentó primero extender y modificar C++ resultando el lenguaje C++ ++ - (++ - porque se añadían y eliminaban características a C++), pero lo abandonó para crear un nuevo lenguaje desde cero al que llamo Oak (roble en inglés, según la versión mas aceptada, por el roble que veía através de la ventana de su despacho).

El resultado fue un lenguaje que tenía similitudes con C, C++ y Objetive C y que no estaba ligado a un tipo de CPU concreta. 
Mas tarde, se cambiaría el nombre de Oak a Java, por cuestiones de propiedad intelectural, al existir ya un lenguaje con el nombre de Oak. Se supone que le pusieron ese nombre mientras tomaban café (Java es nombre de un tipo de café, originario de Asia), aunque otros afirman que el nombre deriva de las siglas de James Gosling, Arthur Van Hoff, y Andy Bechtolsheim.

En Agosto de 1991 Oak ya corría sus primeros programas.

Para 1992, el equipo ya había desarrollado un sistema en un prototipo llamado Star7 (*7), dispositivo parecido a una PDA, cuyo nombre venía de la combinación de teclas del teléfono de la oficina del Proyecto Green que permitía a los usuarios responder al teléfono desde cualquier lugar.

Por su parte, el presidente de la compañía Sun, Scott McNealy, se dio cuenta de forma oportuna y estableció el Proyecto Verde como una subsidiaria de Sun.

Después de mostrar a Scott McNealy y Bill Joy los prototipos de bajo nivel del sistema, continuán con el desarrollo, incluyendo sistema operativo, Green OS; el lenguaje Oak, las librerías, alguna aplicación básica y el hardware, hasta que el 3 de septiembre de 1992 se termina el desarrollo y con ello el Proyecto Verde.

De 1993 a 1994, el equipo de Naughton se lanzó en busca de nuevas oportunidades en el mercado, mismas que se fueron dando mediante el sistema operativo base.
La incipiente subsidiaria fracasó en sus intentos de ganar una oferta con Time-Warner, sin embargo el equipo concluyó que el mercado para consumidores electrónicos smart y las cajas Set-Up en particular, no eran del todo eficaces. La subsidiaria Proyecto verde fue amortizada por la compañía Sun a mediados de 1994.

Afortunadamente, el cese del Proyecto Verde coincidió con el nacimiento del fenómeno mundial WEB. Al examinar las dinámicas de Internet, lo realizado por el ex equipo verde se adecuaba a este nuevo ambiente.

Patrick Naughton procedió a la construcción del lenguaje de programación Java que se accionaba con un browser prototipo. El 29 de septiembre de 1994 se termina el desarrollo del prototipo de HotJava. Cuando se hace la demostración a los ejecutivos de Sun, esta vez, se reconoce el potencial de Java y se acepta el proyecto.

Con el paso del tiempo HotJava se convirtió en un concepto práctico dentro del lenguaje Java y demostró que podría proporcionar multiplataformas para que el código pueda ser bajado y corrido del Host del World Wide Web y que de otra forma no son seguros. 
Una de las características de HotJava fue su soporte para los "applets", que son las partes de Java que pueden ser cargadas mediante una red de trabajo para después ejecutarlo localmente y así lograr soluciones dinámicas en computación acordes al rápido crecimiento del ambiente WEB.

El 23 de mayo de 1995, en la conferencia SunWorld `95, John Gage, de Sun Microsystems, y Marc Andreessen, cofundador y vicepresidente de Netscape, anunciaban la versión alpha de Java, que en ese momento solo corría en Solaris, y el hecho de que Java iba a ser incorporado en Netscape Navigator, el navegador mas utilizado de Internet.

Con la segunda alpha de Java en Julio, se añade el soporte para Windows NT y en la tercera, en Agosto, para Windows 95. 
En enero de 1995 Sun formá la empresa Java Soft para dedicarse al desarrollo de productos basados en la tecnologías Java, y así trabajar con terceras partes para crear aplicaciones, herramientas, sistemas de plataforma y servicios para aumentar las capacidades del lenguaje. Ese mismo mes aparece la versión 1.0 del JDK.

Netscape Communications decide apoyar a Java applets en Netscape Navigator 2.0. Ese fue el factor clave que lanzó a Java a ser conocido y famoso. 
Y como parte de su estrategia de crecimiento mundial y para favorecer la promoción de la nueva tecnología, Java Soft otorgó permisos para otras compañías para que pudieran tener acceso al código fuente y al mismo tiempo mejorar sus navegadores. 
También les permitía crear herramientas de desarrollo para programación Java y los facultaba para acondicionar máquinas virtuales Java (JVM), a varios sistemas operativos.

Muy pronto las licencias o permisos contemplaban prestigiosas firmas como: IBM, Microsoft, Symantec, Silicon Graphics, Oracle, Toshiba y Novell.

Los apples Java (basados en JDK 1.02) son apoyados por los dos más populares navegadores web (Nestcape Navigator 3.0 y Microsoft Internet Explorer 3.0. I.B.M./Lotus, Computer Asociates, Symantec, Informix, Oracle, Sybase y otras poderosas empresas de software están construyendo Software 100 puro JAVA, por ejemplo el Corel Office que actualmente está en versión Beta.

Los nuevos proyectos de Java son co-patrocinados por cientos de millones de dólares en capital disponible de recursos tales como la Fundación Java, un fondo común de capital formado el verano pasado por 11 compañías, incluyendo Cisco Systems, IBM, Netscape y Oracle.

Hoy en día, puede encontrar la tecnología Java en redes y dispositivos que comprenden desde Internet y superordenadores cientifícos hasta portátiles y teléfonos móviles; desde simuladores de mercado en Wall Street hasta juegos de uso doméstico y tarjetas de crédito: Java está en todas partes.



\end{multicols}
\end{document}