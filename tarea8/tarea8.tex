\documentclass[11pt,a4paper]{article}
\usepackage[utf8]{inputenc}
\usepackage[activeacute,spanish]{babel}
\usepackage{graphicx}
\author{Miguel Angel Aguirre Olvera}
\title{Tarea 8 hackers famosos }
\usepackage{multicol}
\DeclareGraphicsExtensions{.png}
\begin{document}
\maketitle 
\begin{multicols}{2}
\begin{center}
\textbf{kevin mitnik}
\end{center}
Kevin Mitnick (1963), nacido en Los Ángeles, comenzó sus primeros pasos sin lo que propiamente haría un hacker: sin un ordenador. En vez de eso, empezó por empaparse del comportamiento de los empleados de determinados sectores para extraer información valiosa. A la corta edad de 13 años supo cómo engatusar a un conductor de autobús para desarrollar una réplica de los billetes para viajar gratis. No fue hasta años después que su destreza en informática tomaría como punto de partida el pirateo a las redes telefónicas, técnica de moda en la época más conocida como «phreaking».

A principios de los años 80 fue acusado de robar manuales privados a la compañía telefónica Pacific Bell y de piratear el sistema NORAD (por sus siglas en inglés: Comando de Defensa Aeroespacial de Norteamérica). Esta «hazaña» le llevó a convertirse en fuente de inspiración de la película Juegos de Guerra (1983). Por lo que no es de extrañar que en esa década fuera apodado por el departamento de Justicia como el «delincuente informático más buscado en la historia de Estados Unidos».

Su particular juego por intentar acceder a cualquier sistema, del que no obtenía nada más que la mera satisfacción de haberlo logrado, se convirtió en una obsesión retribuida con más de una estancia en la cárcel. Tras hackear a entidades como Nokia, Motorola o el Pentágono, fue condenado a cinco años de cárcel. Sin embargo, nunca se refirió a sus actividades como piratería informática, si no como «ingeniería social» por lo que fue el primero en acuñar este término.

Tiempo después decidió abandonar sus malos hábitos, colgó el sombrero negro y se puso el blanco, y se pasó al lado de la ciberseguridad montando su propia empresa. O eso era lo que parecía, porque en 2014 lanzó un portal de venta de herramientas «exploits» de software crítico poco seguras (sin parchear) a cambio de ingentes cantidades de dinero.

Lo cierto es que durante muchos años, supuestamente se le relacionó con más de un delito que no hizo por el desconocimiento de la Justicia en temas de ciberseguridad. En su autobiografía «Un fantasma en el sistema» admite más de una fechoría, a la par que desmiente otras. Al menos, según su versión.
\begin{center}
\textbf{Adrian Lamo}
\end{center}
Adrian Lamo (1981) nació en Boston, pero nunca perteneció a ninguna parte. Fue apodado como «el hacker sin hogar» porque solía deambular sin rumbo y usaba las cafeterías y las bibliotecas como base para sus fechorías. Lo que exponía en menor grado a ser detectado al emplear una red wifi diferente. Años después, descubriría que su actitud disocial estaba relacionada porque tenía asperger.

Este hacker comenzó por pequeñas «jugarretas» como usar una herramienta de administración de contenidos de Yahoo no protegida para modificar un artículo de «Reuters». Y así, poner en evidencia al ex Fiscal General de los Estados Unidos John Ashcroft, tergiversando sus palabras. Pirateaba sistemas y luego lo notificaba a los usuarios de ese servicio o portal y a la prensa. Por lo que, actuaba en cierta manera como un investigador independiente.

Sin embargo, su buena voluntad acabó cuando hackeó la intranet de «The New York Times» para incluirse en una lista de fuentes expertas en piratería informática y recabar información de personajes públicos de gran importancia. También pirateó a otras empresas como Microsoft y el Banco de América.

Su último periplo fue el de entregar a la ex soldado y analista de inteligencia de ejército de Estados Unidos Chelsea Manning por filtrar documentos clasificados. Entre otras cosas, presuntamente entre el material filtrado había entregado a WikiLeaks un vídeo en el que mostraba a soldados estadounidenses asesinando a un fotógrafo de «Reuters» y a otros civiles en Afganistán.

Adrian Lamo falleció el pasado marzo de este 2018 por causas que aún no se han hecho públicas. Aunque la Policía local de la ciudad donde se halló el cadáver en Wichita (Kansas) no sospechaba de que hubiera sido asesinado, lo cierto es que recibía constantemente amenazas.

\begin{center}
\textbf{kevin poulsen}
\end{center}
El apodado hacker «Dark Dante» (1965) pirateó con 17 años la red del Pentágono (Arpanet), ante lo cual se libró de la cárcel porque al ser menor de edad prefirieron castigarlo con una advertencia y no un juicio. Dicho escarmiento no frenó a este joven nacido en Pasadena (California), ya que las autoridades sospecharon que había generado una base de datos de información de una investigación cerrada sobre el ex presidente del Filipinas Ferdinand Marcos.

Tras ver lo que se le venía encima, tuvo que pasar a la clandestinidad durante 17 meses. Momento en el que aprovechó para piratear la red telefónica de un programa de radio para asegurarse ser el ganador de un Porsche. Además, Poulsen se burló del FBI pirateando varios ordenadores federales y revelando detalles de escuchas telefónicas de consulados extranjeros. Por lo que recibió un segundo apodo como «el Hannibal Lecter de los delitos informáticos», por las numerosas leyes que había contravenido.

Sin que le pareciera poco ser un fugitivo, durante un programa de televisión que le mostraba en pantalla como uno de los hackers más buscados, en el momento en que apareció el teléfono para que cualquier espectador aportara cualquier información, las líneas telefónicas se cancelaron en ese instante.

Finalmente, Kevin Poulsen fue capturado y condenado a tres años de prisión, y a no acercarse durante una temporada a ningún ordenador. Cuando salió de la cárcel en 1995 cambió la informática por el periodismo y actualmente se desempeña como editor principal de «Wired» y escribe sobre informática. Aunque no abandonó del todo su amor por los ordenadores, porque en 2006 llevó a las fuerzas del orden ante más de 700 depredadores sexuales que merodeaban por MySpace en busca de menores.

\begin{center}
\textbf{Albert Gonzales}
\end{center}
En el instituto era considerado un líder de «frikis» de la informática, los que le valoraban como tal no sabían cuán lejos eran capaces de llegar sus habilidades. Se hizo asiduo a un portal de comercio criminal llamado «Shadowcrew.com». Sin embargo a los 22 años fue arrestado en Nueva York por fraude con tarjetas relacionado con el robo de datos de millones de cuentas.

Para evitar ir a la cárcel le ofrecieron un trato. Trabajar como informante del servicio secreto del sitio web en el que se había labrado un nombre y una reputación. Las fuerzas del orden llevaban tiempo detrás de procesar a los integrantes de ese servicio. A cambio, recibía la notoria compensación económica al año de 75.000 dólares.

Sin embargo, la avaricia le pudo más. 75.000 dólares le eran insignificantes en comparación de todo el dinero que podía obtener con dichas actividades ilícitas. Por lo que empezó a ejercer un doble papel dentro de ese portal, como informante y como delincuente.

En 2010 fue sentenciado a 20 años de prisión después de confesar haber robado millones de cuentas personales de tarjetas de crédito y débito. Fue considerado el mayor caso de delito informático de Estados Unidos. Actualmente continúa en prisión cumpliendo su pena.

\begin{center}
\textbf{Jonathan James}
\end{center}
Con 15 años Jonathan James (1983), originario de Miami, dejó a Estados Unidos perplejo, porque consiguió entre otras cosas acceder a los servidores de la NASA, descargarse el código fuente de la Estación Espacial Internacional, piratear al departamento de Defensa y al Centro Marshall para Vuelos Espaciales

Tras ser descubierto, «C0mrade», como también era conocido, se convirtió en el primer menor de edad de Estados Unidos en ser condenado por piratería informática a la edad de 16. Finalmente, su condena fue indulgente. Fue condenado a seis meses de arresto domiciliario y libertad condicional hasta cumplir los 18 años, momento en el que falleció su madre. A pesar de que no haber sido ordenado su ingreso en prisión, fue enviado a la cárcel seis meses por dar positivo en drogas.

En 2008, el servicio secreto entró por la fuerza en su casa porque sospechaban que estaba involucrado en el robo de millones de tarjeta de crédito, investigación vinculada a otro pirata informático, Albert Gonzáez, y al portal de venta «Shadowcrew.com». Cierto es que James admitió conocer a algunos de los hacker involucrados en el caso «TJX Hacker», porque era un «mundillo» pequeño, pero siempre negó tener nada que ver con los delitos.

Para este joven, la piratería era un reto del que parecía no querer obtener ninguna retribución económica. Sus actividades no generaban lucro y tenía muy poco dinero. Sin embargo, James empezó a tener miedo de ser usado como chivo expiatorio. Lo que tristemente le llevó a suicidarse el 18 de mayo de 2008 a los 24 años. Antes de morir dejó una carta alegando que no «confiaba en la justicia», como asegura el «Daily Mail».

\begin{center}
\textbf{conclusion}
\end{center}
  puedo concluir  que para ser hacker se necesita mas que ser un buen programador se necesita saber de redes y la estructura de la empresa a la cual se va a hackear ya que sin esa informacion solo estarias intentando entrar a una computadora pero no se podria saber a cual ni nada por el estilo se necesita muchos conocimientos tecnicos en computacion 


\end{multicols}
\end{document}