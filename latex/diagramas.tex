\documentclass[11pt,a4paper]{article}
\usepackage[latin1]{inputenc}
\usepackage[spanish]{babel}
\usepackage{graphicx}
\title{Diagrama de flujo}
\author{Miguel Aguirre}
\usepackage{multicol}
\DeclareGraphicsExtensions{.png}
\begin{document}
\maketitle
\begin{multicols}{2}
\begin{center}
\textbf{Diagrama de flujo}
\end{center}
Un diagrama de fllujo es un diagrama que describe un proceso, sistema o algoritmo informatico. Se usan ampliamente en numerososo campos para documentar, estudiar, planificar, mejorar y comunicar procesos que suelen ser complejos en diagramas claros y faciles de comprender. Los diagramas de flujo emplean rectangulos, ovalos, diamantes y otras numerosas figuras para definir el tipo de paso junto con las flechas conectoras que establecen el flujo y la secuencia. Pueden variar desde diagramas simples dibujados a mano hasta diagramas exhaustivos creados por computadora que describen muchos pasos y rutas.
Si tomamos en cuenta todas las diversas figuras de los diagramas, es uno de los diagramas mas comunes en el mundo 	
\includegraphics[scale=.12]{../../../../Descargas/DiagramaFlujoLampara.png} 
\begin{center}
\textbf{Diagrama NSD}
\end{center}
Basada en un disenio top down de lo complejo a lo simple el problema que se debe resolver se divide en subproblemas cada vez mas pequenio y simples hasta que solo queden instrucciones simples y construcciones para el control de flujo. el diagrama Nassi Shneiderman refleja la descomposicion del problema de una forma simple cuando cajas anidadas para presentar cada uno de los subproblemas para mantener una consistencia con los fundamentos de la programacion estructuradas, los diagramas Nassi Schneiderman no tienen representacion para las instrucciones GOTO 
los NSD son isomericos con los diagramas de flujo. Todo lo que se puede representar con un diagrama de flujo se puede representar en DNS las unicas excepciones son las instrucciones GOTO, break y continue 
\includegraphics[scale=1]{../../../../Descargas/nsd.jpg} 




\end{multicols}
\end{document}