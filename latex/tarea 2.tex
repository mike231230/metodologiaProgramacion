\documentclass[10pt,a4paper]{article}
\usepackage[utf8]{inputenc}
\usepackage[activeacute,spanish]{babel}
\author{Miguel Angel Aguirre Olvera}
\title{Tarea 2 Generaciones de computadoras}
\usepackage{multicol}


\begin{document}
\maketitle

\begin{multicols}{2}
\begin{center}
\textbf{Primera generacion 1951-1958}
\end{center}
Las computadoras de la primera Generación emplearon bulbos para procesar información. Los operadores ingresaban los datos y programas en código especial por medio de tarjetas perforadas. El almacenamiento interno se lograba con un tambor qEl invento del transistor hizo posible una nueva generación de computadoras, más rápidas, más pequeñas y con menores necesidades de ventilación. Sin embargo el costo seguía siendo una porción significativa del presupuesto de una Compañía. Las computadoras de la segunda generación también utilizaban redes de núcleos magnéticos en lugar de tambores giratorios para el almacenamiento primario. Estos núcleos contenían pequeños anillos de material magnético, enlazados entre sí, en los cuales podían almacenarse datos e instrucciones.
Los programas de computadoras también mejoraron. El COBOL (COmmon Busines Oriented Languaje) desarrollado durante la 1era generación estaba ya disponible comercialmente, este representa uno de los mas grandes avances en cuanto a portabilidad de programas entre diferentes computadoras; es decir, es uno de los primeros programas que se pueden ejecutar en diversos equipos de computo después de un sencillo procesamiento de compilación. Los programas escritos para una computadora podían transferirse a otra con un mínimo esfuerzo. Grace Murria Hooper (1906-1992), quien en 1952 habia inventado el primer compilador fue una de las principales figuras de CODASYL (Comité on Data SYstems Languages), que se encago de desarrollar el proyecto COBOL. El escribir un programa ya no requería entender plenamente el hardware de la computación. Las computadoras de la 2da Generación eran sustancialmente más pequeñas y rápidas que las de bulbos, y se usaban para nuevas aplicaciones, como en los sistemas para reservación en líneas aéreas, control de tráfico aéreo y simulaciones para uso general. Las empresas comenzaron a aplicar las computadoras a tareas de almacenamiento de registros, como manejo de inventarios, nómina y contabilidad.
 
 \begin{center}
\textbf{Segunda generacion de computadoras 1959-1964 }
\end{center} 
El invento del transistor hizo posible una nueva generación de computadoras, más rápidas, más pequeñas y con menores necesidades de ventilación. Sin embargo el costo seguía siendo una porción significativa del presupuesto de una Compañía. Las computadoras de la segunda generación también utilizaban redes de núcleos magnéticos en lugar de tambores giratorios para el almacenamiento primario. Estos núcleos contenían pequeños anillos de material magnético, enlazados entre sí, en los cuales poEl invento del transistor hizo posible una nueva generación de computadoras, más rápidas, más pequeñas y con menores necesidades de ventilación. Sin embargo el costo seguía siendo una porción significativa del presupuesto de una Compañía. Las computadoras de la segunda generación también utilizaban redes de núcleos magnéticos en lugar de tambores giratorios para el almacenamiento primario. Estos núcleos contenían pequeños anillos de material magnético, enlazados entre sí, en los cuales podían almacenarse datos e instrucciones.
Los programas de computadoras también mejoraron. El COBOL (COmmon Busines Oriented Languaje) desarrollado durante la 1era generación estaba ya disponible comercialmente, este representa uno de los mas grandes avances en cuanto a portabilidad de programas entre diferentes computadoras; es decir, es uno de los primeros programas que se pueden ejecutar en diversos equipos de computo después de un sencillo procesamiento de compilación. Los programas escritos para una computadora podían transferirse a otra con un mínimo esfuerzo. Grace Murria Hooper (1906-1992), quien en 1952 habia inventado el primer compilador fue una de las principales figuras de CODASYL (Comité on Data SYstems Languages), que se encago de desarrollar el proyecto COBOL. El escribir un programa ya no requería entender plenamente el hardware de la computación. Las computadoras de la 2da Generación eran sustancialmente más pequeñas y rápidas que las de bulbos, y se usaban para nuevas aplicaciones, como en los sistemas para reservación en líneas aéreas, control de tráfico aéreo y simulaciones para uso general. Las empresas comenzaron a aplicar las computadoras a tareas de almacenamiento de registros, como manejo de inventarios, nómina y contabilidad.
 
\begin{center}
\textbf{Tercera generacion de computadoras 1964-1971}
\end{center}
Las computadoras de la tercera generación emergieron con el desarrollo de los circuitos integrados (pastillas de silicio) en las cuales se colocan miles de componentes electrónicos, en una integración en miniatura. Las computadoras nuevamente se hicieron más pequeñas, más rápidas, desprendían menos calor y eran energéticamente más eficientes.
El descubrimiento en 1958 del primer Circuito Integrado (Chip) por el ingeniero Jack S. Kilby (nacido en 1928) de Texas Instruments, así como los trabajos que realizaba, por su parte, el Dr. Robert Noyce de Fairchild Semicon ductors, acerca de los circuitos integrados, dieron origen a la tercera generación de computadoras.
Antes del advenimiento de los circuitos integrados, las computadoras estaban diseñadas para aplicaciones matemáticas o de negocios, pero no para las dos cosas. Los circuitos integrados permitieron a los fabricantes de computadoras incrementar la flexibilidad de los programas, y estandarizar sus modelos.
La IBM 360 una de las primeras computadoras comerciales que usó circuitos integrados, podía realizar tanto análisis numéricos como administración ó procesamiento de archivos.
IBM marca el inicio de esta generación, cuando el 7 de abril de 1964 presenta la impresionante IBM 360, con su tecnología SLT (Solid Logic Technology). Esta máquina causó tal impacto en el mundo de la computación que se fabricaron más de
30000, al grado que IBM llegó a conocerse como sinónimo de computación.
También en ese año, Control Data Corporation presenta la supercomputadora CDC 6600, que se consideró como la más poderosa de las computadoras de la época, ya que tenía la capacidad de ejecutar unos 3 000 000 de instrucciones por segundo (mips).
Se empiezan a utilizar los medios magnéticos de almacenamiento, como cintas magnéticas de 9 canales, enormes discos rígidos, etc. Algunos sistemas todavía usan las tarjetas perforadas para la entrada de datos, pero las lectoras de tarjetas ya alcanzan velocidades respetables.
Los clientes podían escalar sus sistemas 360 a modelos IBM de mayor tamaño y podían todavía correr sus programas actuales. Las computadoras trabajaban a tal velocidad que proporcionaban la capacidad de correr más de un programa de manera simultánea (multiprogramación).
\begin{center}
\textbf{Cuarta generacion de computaadoras 1971-1981}
\end{center}
Dos mejoras en la tecnología de las computadoras marcan el inicio de la cuarta generación: el reemplazo de las memorias con núcleos magnéticos, por las de chips de silicio y la colocación de muchos más componentes en un Chip: producto de la microminiaturización de los circuitos electrónicos. El tamaño reducido del microprocesador y de chips hizo posible la creación de las computadoras personales (PC)
En 1971, intel Corporation, que era una pequeña compañía fabricante de semiconductores ubicada en Silicon Valley, presenta el primer microprocesador o Chip de 4 bits, que en un espacio de aproximadamente 4 x 5 mm contenía 2250 transistores. Este primer microprocesador que se muestra en la figura 1.14, fue bautizado como el 4004.
Silicon Valley (Valle del Silicio) era una región agrícola al sur de la bahía de San Francisco, que por su gran producción de silicio, a partir de 1960 se convierte en una zona totalmente industrializada donde se asienta una gran cantidad de empresas fabricantes de semiconductores y microprocesadores. Actualmente es conocida en todo el mundo como la región más importante para las industrias relativas a la computación: creación de programas y fabricación de componentes.
\begin{center}
\textbf{Quinta generacion de computadoras 1982-1999}
\end{center}
Cada vez se hace más difícil la identificación de las generaciones de computadoras, porque los grandes avances y nuevos descubrimientos ya no nos sorprenden como sucedió a mediados del siglo XX. Hay quienes consideran que la cuarta y quinta generación han terminado, y las ubican entre los años 1971-1984 la cuarta, y entre 1984-1990 la quinta. Ellos consideran que la sexta generación está en desarrollo desde 1990 hasta la fecha.

Siguiendo la pista a los acontecimientos tecnológicos en materia de computación e informática, podemos puntualizar algunas fechas y características de lo que podría ser la quinta generación de computadoras.

Con base en los grandes acontecimientos tecnológicos en materia de microelectrónica y computación (software) como CADI CAM, CAE, CASE, inteligencia artificial, sistemas expertos, redes neuronales, teoría del caos, algoritmos genéticos, fibras ópticas, telecomunicaciones, etc., a de la década de los años ochenta se establecieron las bases de lo que se puede conocer como quinta generación de computadoras.

Hay que mencionar dos grandes avances tecnológicos, que sirvan como parámetro para el inicio de dicha generación: la creación en 1982 de la primera supercomputadora con capacidad de proceso paralelo, diseñada por Seymouy Cray, quien ya experimentaba desde 1968 con supercomputadoras, y que funda en 1976 la Cray Research Inc.; y el anuncio por parte del gobierno japonéCada vez se hace más difícil la identificación de las generaciones de computadoras, porque los grandes avances y nuevos descubrimientos ya no nos sorprenden como sucedió a mediados del siglo XX. Hay quienes consideran que la cuarta y quinta generación han terminado, y las ubican entre los años 1971-1984 la cuarta, y entre 1984-1990 la quinta. Ellos consideran que la sexta generación está en desarrollo desde 1990 hasta la fecha.

Siguiendo la pista a los acontecimientos tecnológicos en materia de computación e informática, podemos puntualizar algunas fechas y características de lo que podría ser la quinta generación de computadoras.

Con base en los grandes acontecimientos tecnológicos en materia de microelectrónica y computación (software) como CADI CAM, CAE, CASE, inteligencia artificial, sistemas expertos, redes neuronales, teoría del caos, algoritmos genéticos, fibras ópticas, telecomunicaciones, etc., a de la década de los años ochenta se establecieron las bases de lo que se puede conocer como quinta generación de computadoras.

Hay que mencionar dos grandes avances tecnológicos, que sirvan como parámetro para el inicio de dicha generación: la creación en 1982 de la primera supercoCada vez se hace más difícil la identificación de las generaciones de computadoras, porque los grandes avances y nuevos descubrimientos ya no nos sorprenden como sucedió a mediados del siglo XX. Hay quienes consideran que la cuarta y quinta generación han terminado, y las ubican entre los años 1971-1984 la cuarta, y entre 1984-1990 la quinta. Ellos consideran que la sexta generación está en desarrollo desde 1990 hasta la fecha.

Siguiendo la pista a los acontecimientos tecnológicos en materia de computación e informática, podemos puntualizar algunas fechas y características de lo que podría ser la quinta generación de computadoras.

Con base en los grandes acontecimientos tecnológicos en materia de microelectrónica y computación (software) como CADI CAM, CAE, CASE, inteligencia artificial, sistemas expertos, redes neuronales, teoría del caos, algoritmos genéticos, fibras ópticas, telecomunicaciones, etc., a de la década de los años ochenta se establecieron las bases de lo que se puede conocer como quinta generación de computadoras.

Hay que mencionar dos grandes avances tecnológicos, que sirvan como parámetro para el inicio de dicha generación: la creación en 1982 de la primera supercomputadora con capacidad de proceso paralelo, diseñada por Seymouy Cray, quien ya experimentaba desde 1968 con supercomputadoras, y que funda en 1976 la Cray Research Inc.; y el anuncio por parte del gobierno japonés del proyecto "quinta generación", que según se estableció en el acuerdo con seis de las más grandes empresas japonesas de computación, debería terminar en 1992.mputadora con capacidad de proceso paralelo, diseñada por Seymouy Cray, quien ya experimentaba desde 1968 con supercomputadoras, y que funda en 1976 la Cray Research Inc.; y el anuncio por parte del gobierno japonés del proyecto "quinta generación", que según se estableció en el acuerdo con seis de las más grandes empresas japonesas de computación, debería terminar en 1992.s del proyecto "quinta generación", que según se estableció en el acuerdo con seis de las más grandes empresas japonesas de computación, debería terminar en 1992.
\begin{center}
\textbf{bibliografia}
\end{center}
Victor F. Generaciones de las computadoras[online]
sites.google.com/site/is23generaciones/quinta-generacion-1982-1989 
\end{multicols}
\end{document}
