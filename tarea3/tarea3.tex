\documentclass[11pt,a4paper]{article}
\usepackage[utf8]{inputenc}
\usepackage[activeacute,spanish]{babel}
\usepackage{graphicx}
\author{Miguel Angel Aguirre Olvera}
\title{Tarea 3 Logaritmos}
\usepackage{multicol}
\DeclareGraphicsExtensions{.png}


\begin{document} 
\maketitle 
\begin{multicols}{2}
\begin{center}
\textbf{Problema 1}
\end{center}
Pseudo codigo

Inicio

var

ent x,y,z

Imprimir "dame 3 numeros"

almacenar

numero 1=x

numero 2=y

numero 3=z

fin alamcenar 

si x+y=z

entonces imprimir "iguales"

si z+y=x

entonces imprimir "iguales"

si z+x=y 
entonces imprimir "iguales"

sino imprimir "diferentes"

 
fin  

\textbf{diagrama de flujo}

\includegraphics[scale=.50]{diagramap1.png} 



\begin{center}

\textbf{Problema 2}

\end{center}
\textbf{Pseudo codigo}

inicio

var 

Entero= a,b,c,d

imprimir "dame 4 numeros"

almacenar 
numero1=a
numero2=b
numero3=c
numero4=d


si a,b,c,d

entonces

si a>b y 

si a>c y 

si a>d 

imprimir a

sino 

si b>c

si b>d entonces imrimir b

sino 

si c>d entonces imprimir c

sino 

entonces imprimir d
 
fin si 
 

\textbf{diagrama de flujo}



\includegraphics[scale=.12]{diagrama2.png} 








\begin{center}

\textbf{problema 3}
\end{center}
\textbf{pseudo codigo}

inicio

var ent= a,b,c

imprimir "dame a"

almacenar a

imprimir "dame b"

almacenar b

si

a y b entonces $a^2+b^2= c^2$

imprimir c

fin 

\textbf{diagrama de flujo}

 

\includegraphics[scale=.5]{diagramap3.png} 







\begin{center}

\textbf{problema 4}
\end{center}
\textbf{pseudo codigo}


inicio

var ent= c,td,tf

 dame costo

 c=costo
 
 td=c*.80

 tf= td*1.15
 
 imprimir td,tf

  fin 




\textbf{diagrama de flujo}





\includegraphics[scale=.5]{diagramap4.png} 
\end{multicols}
 
\end{document}