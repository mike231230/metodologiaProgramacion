\documentclass[10pt,a4paper]{article}
\usepackage[utf8]{inputenc}
\usepackage[activeacute,spanish]{babel}
\usepackage{graphicx}
\author{Miguel Angel Aguirre Olvera}
\title{Practica 1}
\usepackage{multicol}
\DeclareGraphicsExtensions{.png}

\begin{document}
\maketitle

\begin{multicols}{2}
realizar los siguientes algoritmos y diagramas de flujo y NS
\begin{center}
\textbf{Problema 1}  
\end{center}
a) realizar una llamada telefonica

pseudocodigo:

inicio
Escribir 'desbloquea el telefono'

	Escribir 'abrir aplicacion de llamada'
	
	Escribir 'consulta tu saldo'
	
	Si tengo saldo Entonces
	
		Escribir 'teclea el numero telefonico'
		
		Si contesta Entonces
		
			Escribir 'puedes hablar'
			
		SiNo
			Escribir 'llamada transferida al buzon'
			
			Si dejas mensaje Entonces
			
				Escribir 'no colgar y dejar mensaje'
				
			SiNo
				Escribir 'colgar'
			FinSi
			
			Si terminar Entonces
			
				Escribir 'colgar'

			FinSi

		FinSi

	SiNo

		Escribir "no puedes realizar la llamada"

	FinSi

Fin

diagrama de flujo:


\includegraphics[scale=.18]{llamada .png} 

diagrama de NS:

\includegraphics[scale=.17]{llamadans .png} 


b) cocinar una tortilla 

pseudocodigo 
inicio 

Escribir 'verificar que tengas tortillas'

	Escribir 'verificar que tengas gas'

	Si nohaygas Entonces

		Escribir 'comprar tortillas'

	SiNo

		Escribir 'enciende la estufa'

		Escribir 'colocar comal'

		Escribir 'colocar a fuego lento'

		Escribir 'colocar la tortilla'

		Escribir 'voltear cada 2 minutos'
		
		Escribir 'sacar la tortilla y colocarla en cervilleta'
	
	FinSi
	
fin 

diagrama de flujo

\includegraphics[scale=.18]{tortillas.png} 
	
diagrama ns 

\includegraphics[scale=.15]{tortillasns.png} 

c)cambiar una llanta ponchada

pseudocodigo: 

inicio

Escribir 'bajar del auto'

	Escribir 'abir la cajuela'

	Si hayherramienta Entonces

		Escribir 'bajar la herramienta'

		Escribir 'bajar la llanta de refaccion'

		Escribir 'colocar el gato debajo del auto'

		Escribir 'subir con el gato el auto'

		Escribir 'quitar los pernos de la llanta'

		Escribir 'retirar la llanta'

		Escribir 'colocar la llanta de refaccion'

		Escribir 'colarla de manera que quede igual con los hoyos'

		Escribir 'apretar los pernos a la nueva llanta'

		Escribir 'bajar el auto con el gato'

		Escribir 'guardar la herramienta'

	SiNo

		Escribir 'llama a un mecanico o grua'

	FinSi

Fin

	diagrama de flujo 

\includegraphics[scale=.18]{cambiollanta.png} 	

diagrama ns:
 
\includegraphics[scale=.18]{cambiollantans.png} 


D)freir un huevo 

inicio

Escribir 'abrir el refrigerador'

	Si hayhuevos Entonces

		Escribir 'sacarlos del refrigerador'

		Escribir 'ir a la cocina'

		Si haygas Entonces

			Escribir 'prender la estufa'

			Si hayaceite Entonces

				Escribir 'seguir a lo siguiente'

			SiNo

				Escribir 'comprar aceite '

			FinSi

			Escribir 'colocar aceite en el sarten'

			Escribir 'romper el huevo'

			Escribir 'esperar a que se fria '

			Escribir 'retirar el huevo'

			Escribir 'colocarlo en un plato'

			Escribir 'apagar estufa'

		SiNo

			Escribir 'ir a un restaurante'

		FinSi

	SiNo

		Escribir ' comprar algo en la tienda preparado'

	FinSi

Fin

diagrama de flujo:
 
\includegraphics[scale=.1]{huevofrito.png}

diagrma ns: 

\includegraphics[scale=.1]{huevofritons.png}  

E) hacer quesadilla 

inicio

Escribir 'verificar si tengo tortillas'

	Escribir 'verificar si tengo queso o ingredientes necesarios'

	Escribir 'calentar un comal a fuego lento'

	Escribir 'colocar la tortilla y calentar 20 seg'

	Escribir 'sacar la tortilla y colocarla en el plato'

	Escribir 'tomar el queso y rellenar la tortilla'

	Escribir 'colocar de nuevo la tortilla en el comal'

	Si quesoderretido Entonces

		Escribir 'sacar la quesadilla del comal cuando el queso este 
		derretido '

	SiNo

		Escribir 'voltear la tortilla cada 2 min hasta que se derrita el 
		queso'

	FinSi

Fin

diagrama de flujo 

\includegraphics[scale=.18]{quesadilla.png} 


diagrama ns

\includegraphics[scale=.15]{quesadillans.png}

\begin{center}
\textbf{ejercicio 2}
\end{center}

algoritmo suma resta multiplicacion y division 

inicio 

	Escribir 'dame un numero'

	Leer a

	Escribir 'dame un numero'

	Leer b

	suma <- a+b

	resta <- a-b

	multiplicacion <- a*b

	division <- a/b

	Escribir 'la suma es ',suma

	Escribir 'la resta es: ',resta

	Escribir 'la multiplicacion es: ',multiplicacion
	
	Escribir 'la diviion es: ',division

Fin

diagrama de flujo 
 
\includegraphics[scale=.18]{sum-div-mult-resf.png} 
	
diagrama ns

\includegraphics[scale=.18]{sum-div-mult-res.png} 

\begin{center}
\textbf{ejercicio 3}
\end{center}	

maximo comun divisor

pseudocodigo

	inicio
	
	Definir a,b,c,d,res Como Entero

	Escribir 'dame el primer numero'

	Leer a

	Escribir 'dame el segundo numero'

	Leer b

	Si a>b Entonces

		c <- a

		d <- b

	SiNo

		c <- b

		d <- a

	FinSi

	Mientras d!=0 Hacer

		res <- d

		d <- c MOD d

		c <- res

	FinMientras
	
	Escribir 'el mcd entre ',a,' y ',b,' es: ',res

Fin

diagrama de flujo

\includegraphics[scale=.18]{mdc.png} 

diagrama ns:

\includegraphics[scale=.18]{mdcns.png}

\begin{center}
\textbf{problema 4}
\end{center} 

visualizar numeros leidos y terminar en cero

pseudocodigo 

inicio

Definir a,b,c,d,e Como Entero

	Escribir 'dame un numero'

	Leer a

	Escribir 'dame un numero'

	Leer b

	Escribir 'dame otro numero'

	Leer c

	Escribir 'dame el ultimo numero'

	Leer d

	Escribir 'los numeros que me diste son ',a,' ',b,' ',c,' y ',d

	e <- 0

	Si a+b+c+d=e Entonces

	FinSi

Fin

diagrama de flujo 

\includegraphics[scale=.18]{valorcero.png} 

diagrama NS

\includegraphics[scale=.18]{valorcerons.png} 


\begin{center}
\textbf{problema 5}
\end{center}

visualizar de 3 en 3 hasta el 99

inicio

Definir a,b Como Entero

	Repetir

		a <- a+3

		Escribir a

		b <- b+1

	Hasta Que b=33

Fin

diagrama flujo 

\includegraphics[scale=.18]{detresentres.png}

diagrama ns

\includegraphics[scale=.18]{detresentresns.png}  	

\begin{center}
\textbf{problema 6}
\end{center}

numero mayor

pseudocodigo 

inicio

Definir a,b,c,d Como Entero

	Escribir 'dame un numero'

	Leer a

	Escribir 'dame un numero'

	Leer b

	Escribir 'dame un numero'

	Leer c

	Escribir 'dame un numero'

	Leer d

	Si a>b Y a>c Y a>d Entonces

		Escribir 'el mayor es ',a

	FinSi

	Si b>a Y b>c Y b>d Entonces

		Escribir 'el mayor es',b

	FinSi

	Si c>a Y c>b Y c>d Entonces

		Escribir 'el mayor es',c

	FinSi

	Si d>a Y d>b Y d>c Entonces

		Escribir 'el mayor es ',d
	
	FinSi

Fin

diagrama de flujo

\includegraphics[scale=.18]{mayornumero.png} 	
	
diagrama ns

\includegraphics[scale=.18]{mayornumerons.png}

\begin{center}
\textbf{problema 7}
\end{center} 	

tres numeros diferentes y verificar si un numero con otro sumado da igual a cualquier otro numero dado 

inicio

	Definir a,b,c Como Entero

	Escribir 'dame un numero'

	Leer a

	Escribir 'dame otro numero'

	Leer b

	Escribir 'dame otro numero'

	Leer c

	Si a+b=c Entonces

		Escribir a,'+',b,'es igual a ',c

	SiNo

		Si a+c=b Entonces

			Escribir a,'+',c,'es igual a ',b

		SiNo

			Si c+b=a Entonces

				Escribir c,'+',b,'es igual a ',a

			SiNo

				Escribir 'ningun numero es igual a la suma'

			FinSi

		FinSi
	
	FinSi
Fin

diagrama de flujo 

\includegraphics[scale=.1]{abyc.png}

diagrama ns

\includegraphics[scale=.09]{abycns.png}

\begin{center}
\textbf{problema 8}
\end{center}
  	numeros primos 
pseudocodigo

inicio

Escribir 'dame un numero'

	Leer n

	Para x<-2 Hasta n Hacer

		m <- 2

		band <- verdadero

		Mientras band=verdadero Y m<x Hacer

			Si x MOD m=0 Entonces

				band <- Falso

			SiNo

				m <- m+1

			FinSi

		FinMientras

		Si band=verdadero Entonces

			Escribir 'el numero ',x,'es primo'

		FinSi
	
	FinPara
Fin
 
 diagrama de flujo
 
\includegraphics[scale=.18]{primos.png} 

diagrama ns

\includegraphics[scale=.18]{primosns.png} 
	
\begin{center}
\textbf{problema 9}
\end{center}	
	area de un triangulo 
	
	pseudocodigo 
inicio

Escribir 'dame la superficie de la base'

	Leer b

	Escribir 'dame la superficie de la altura'

	Leer h

	a = (b*h)/2
	
	Escribir a
	
Fin

diagrama de flujo 

\includegraphics[scale=.18]{triangulo.png}

diagrama ns

\includegraphics[scale=.18]{triangulons.png}

\begin{center}
\textbf{problema 10}
\end{center}

circunferencia y area de un circulo 

pseudocodigo 

inicio

Escribir 'dame el radio'
	
	Leer r

	circunferencia <- (r*2)*PI

	Escribir 'la circunferencia es ',circunferencia

	area <- (r*r)*PI

	Escribir 'el area es ',area

Fin
  	
diagrama de flujo 

\includegraphics[scale=.18]{circulo.png} 

diagrama ns 

\includegraphics[scale=.18]{circulons.png} 	

\begin{center}
\textbf{problema 11}
\end{center}

salario semanal 

pseudocodigo 

inicio

Definir nombre Como Caracter

	Definir ph,jl,sb,st,isr Como Real

	Escribir 'nombre del trabajador'

	Leer nombre

	Escribir 'horas trabajadas'

	Leer jl

	Escribir 'precio por hora'

	Leer ph

	sb <- jl*ph

	isr <- sb*.10

	st <- sb-isr
	
	Escribir ' el salario total es ',st

Fin

diagrama de flujo 

\includegraphics[scale=.18]{salario.png}

diagrama ns

\includegraphics[scale=.18]{salarions.png}  
	
\end{multicols}
\end{document}